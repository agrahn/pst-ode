\documentclass{article}
\usepackage{pst-ode,pst-3dplot}
\usepackage[T1]{fontenc}
\usepackage[latin1]{inputenc}%

\begin{document}
\begin{center}
\pstVerb{
  /alpha 10 def
  /beta 28 def
  /gamma 8 3 div def
}%
\pstODEsolve[
  algebraic,
%  saveData % Zuschalten der Option schreibt Ergebnis in lorenzXYZ.data
            % erfordert ps2pdf -dNOSAFER 
]{lorenzXYZ}{0 1 2}{0}{25}{2501}{10 10 30}{
  alpha*(x[1]-x[0]) |
  x[0]*(beta-x[2]) - x[1] |
  x[0]*x[1] - gamma*x[2]
}
\fbox{\begin{pspicture}(-6,-2)(6,10)
\psset{unit=0.17cm,Alpha=45,Beta=15}
\listplotThreeD{lorenzXYZ}
\psset{unit=0.425cm,linestyle=dashed}
\pstThreeDNode(0,0,0){O}
\pstThreeDNode(0,0,5){Z}
\pstThreeDNode(5,0,0){X}
\pstThreeDNode(0,5,0){Y}
\pstThreeDNode(-10,-10,0){A}
\pstThreeDNode(-10,-10,20){B}
\pstThreeDNode(-10,10,20){C}
\pstThreeDNode(-10,10,0){D}
\pstThreeDNode(10,-10,0){E}
\pstThreeDNode(10,-10,20){F}
\pstThreeDNode(10,10,20){G}
\pstThreeDNode(10,10,0){H}
\pspolygon(A)(B)(C)(D)
\pspolygon(E)(F)(G)(H)
\psline(A)(E)
\psline(B)(F)
\psline(D)(H)
\psline(C)(G)
\psset{linestyle=solid,linecolor=red}
\psline{->}(O)(X)
\psline{->}(O)(Y)
\psline{->}(O)(Z)
\end{pspicture}}
\end{center}
\end{document}
